\chapter{Project Documentation}

% How does your project work? (Could include the following...)
%   What is its structure?
%   What is its Theory of Operation?
%    Block and flow diagrams are good here.
%  How does one install your software, if any?
%  How does one run it?
%  Are there any special hardware, OS, or runtime requirements to run your software?
%  Any user guides, API documentation, etc.

%This needs to be detailed enough to recreate and/or use your project!

\section{Introduction}

This section lays out how to build, use, and modify our project.
All the source code and supporting documents can be found on Github \cite{Github}.

\section{Building}

To build for the first time, run make.sh in the top-level directory.

\begin{lstlisting}
./make.sh
\end{lstlisting}

This will build all the third party libraries as well as our code.

For subsequent builds, simply run \texttt{make} in the \texttt{src} directory.

\begin{lstlisting}
cd src
make
\end{lstlisting}

\section{Usage}

\subsection{Command-line}

The program is split into two halves: a client and a server.

\subsubsection{Server}

To run the server, simply run

\begin{lstlisting}
./server
\end{lstlisting}

It will sit in the foreground and listen for requests from the client.
By default, the server listens on port 24992.
To listen on another port, pass the port number as the first argument to the server.

\begin{lstlisting}
./server 8000
\end{lstlisting}

\subsubsection{Client}

The client is a little more interesting.
It has a number of subcommands;
to see the full list, give it the \texttt{-h} option (or any other invalid option

\begin{lstlisting}
\% ./client -h
./client: invalid option -- 'h'
Usage: client [-p port] <command> [args]
Commands:
    client setup [files...]
    client search <word>
    client add <fileid> [words...]
    client addfile <filename>
    client delete <fileid> [words...]
\end{lstlisting}

For any of these commands to work, the server must already be running.
If the server is not running, the client will hang until the server is started.

\noindent\textbf{Setup}

The setup command creates an encrypted search index

\begin{lstlisting}
client setup [files...]
\end{lstlisting}

For example,

\begin{lstlisting}
./client setup *.cpp
\end{lstlisting}

will initialize the search index with all cpp files in the current directory.

There is also a hidden \texttt{setuplist} command, which reads the list of files from another file
instead of the command line.

For example,

\begin{lstlisting}
ls *.cpp >filelist
./client setuplist filelist
\end{lstlisting}

This does the same thing as the previous setup command, but can be used even if
the number of files is too large to fit on the command line.

\noindent\textbf{Search}

The search command searches for a keyword in the encrypted index.
It prints out a list of matching files.

\begin{lstlisting}
client search [keyword]
\end{lstlisting}

For example,

\begin{lstlisting}
\% ./client search int
info: loaded client state
got response
int: bench.cpp
int: client.cpp
int: client_test.cpp
int: DSSEClient.cpp
int: DSSE.cpp
int: DSSEServer.cpp
int: fail.cpp
int: merged-client.cpp
int: server.cpp
int: speed.cpp
int: storage.cpp
\end{lstlisting}

\noindent\textbf{Add}

The add command adds one or more keywords to a file.
To use the add command, you need to know the id of the file you want to modify.

\begin{lstlisting}
client add <fileid> [keywords]
\end{lstlisting}

For example,

\begin{lstlisting}
./client add 1 hello
\end{lstlisting}

Now searching for `hello' will return the file with id number 1.

\begin{lstlisting}
\% ./client search hello
info: loaded client state
got response
hello: bench.cpp
\end{lstlisting}

There is also a variant of the add command called addfile.
The addfile command adds an entire new file to the search index.

\begin{lstlisting}
\% ./client addfile DSSE.h
\end{lstlisting}

\noindent\textbf{Delete}

The delete command deletes one or more keywords from a file.
It works similarly to the add command.
To use the delete command, you need to know the id of the file you want to modify.

\begin{lstlisting}
client delete <fileid> [keywords]
\end{lstlisting}

For example,

\begin{lstlisting}
./client delete 1 hello
\end{lstlisting}

\subsection{Demo app}

The demo app lives in the \texttt{demo\_app} branch.
It is not part of the main code base.
The app shells out to the \texttt{client} program for all database operations;
you should compile the main programs and get the server running before continuing.
You will also need to initialize the database with \texttt{client setup} as described above.

The app is written in Python using Flask.
We recommend using a virtualenv to install Flask and other dependencies.

\begin{lstlisting}
git checkout demo_app
cd demo
virtualenv .
./bin/pip install -r requirements.txt
\end{lstlisting}

To run the app, just run the script,

\begin{lstlisting}
./run
\end{lstlisting}

or manually launch python from the virtualenv.

\begin{lstlisting}
./bin/python app.py
\end{lstlisting}

You will probably have to set the \texttt{DB\_PATH} variable in \texttt{app.py}
to point to the directory containing the files in the database.
These are expected to be local to the computer running the demo app.

\begin{lstlisting}
DB_PATH = os.path.expanduser("~/maildir")
\end{lstlisting}

\section{Hacking}

The top level of the repository is layed out as follows.

\begin{center}
\begin{tabular}{ll}
    Benchmarks/         & benchmark results \\
    Meeting\_Notes/      & meeting notes throughout the term \\
    Poster/             & our expo poster \\
    Term\_PowerPoints/   & powerpoint slides from our progress reports \\
    demo/               & (demo\_app branch only) this is where the demo app lives \\
    reports/            & capstone reports and other documents \\
    src/                & the main source code \\
    third\_party/        & copies of third-party libraries that we depend on \\
\end{tabular}
\end{center}

The most interesting directory is \texttt{src}, which contains all the main code for the DSSE.


Within src,

\begin{center}
\begin{tabular}{ll}
    DSSE.h                      & central header file \\
    DSSE.cpp                    & core DSSE class (all cryptographic code) \\
    DSSEClient.cpp              & DSSE client class  (all client networking logic) \\
    DSSEServer.cpp              & DSSE server class  (all server networking logic) \\
    DSSETokenize.cpp            & tokenization code \\
    storage.cpp                 & storage serialization code \\
    \hline
    dsse-prototype.py           & a prototype written in Python \\
    dsse.proto                  & protobuf definitions \\
    \hline
    bench.cpp                   & some benchmarking code \\
    client.cpp                  & client program - uses DSSE::Client \\
    server.cpp                  & server program - uses DSSE::Server \\
\end{tabular}
\end{center}


The DSSE is organized around three central classes ---
\texttt{DSSE::Core}, \texttt{DSSE::Client}, and \texttt{DSSE::Server}.
There is one central header file, \texttt{DSSE.h},
which contains declarations for all these classes and their associated methods.


\texttt{DSSE.cpp} defines the \texttt{DSSE::Core} class, and
contains all the low-level cryptographic code.
All the algorithms described in the paper are implemented in this file.
It contains both the client and server halves of the Setup, Search, Add, and Delete methods.
It does not contain any networking code.
The Core class is also where all the state required by the algorithms are stored.
\texttt{storage.cpp} contains functions which are able to serialize the this state into files.

\texttt{DSSEClient.cpp} and \texttt{DSSEServer.cpp} define the
\texttt{DSSE::Client} and \texttt{DSSE::Server} classes,
which implement the client and server halves of the network communication, respectively.
All the networking is performed in these files.
These classes interact with the \texttt{DSSE::Core} class;
they deserialized messages from the wire, pass them to the core, and serialize the results back out.
Messages are serialized using protobuf; messages definitions can be found in \texttt{dsse.proto}.

\texttt{dsse-prototype.py} contains a prototype implementation of the core DSSE protocol (basic variant).
If you just want to understand the algorithm, this may be a good place to look.

Finally, \texttt{client.cpp} and \texttt{server.cpp} are the
actual client and server programs. This is where the \texttt{main} function is
defined.
For the most part, these are just simple wrappers around the
\texttt{DSSE::Client} and \texttt{DSSE::Server} classes;
they parse command line arguments, construct an object of the appropriate class,
call one or more methods, and print the results.
\texttt{bench.cpp} is the program we used for some of the benchmarks.

Further details and documentation of the classes can be found in \texttt{DSSE.h}.

\chapter{Project Documentation}

% How does your project work? (Could include the following...)
%   What is its structure?
%   What is its Theory of Operation?
%    Block and flow diagrams are good here.
%  How does one install your software, if any?
%  How does one run it?
%  Are there any special hardware, OS, or runtime requirements to run your software?
%  Any user guides, API documentation, etc.

%This needs to be detailed enough to recreate and/or use your project!

\section{Introduction}

This section lays out how to build, use, and modify our project.
All the source code and supporting documents can be found on Github \cite{Github}.

\section{Building}

To build for the first time, run make.sh in the top-level directory.

    ./make.sh

This will build all the third party libraries as well as our code.

For subsequent builds, simply run `make` in the `src` directory.

    cd src
    make

\section{Usage}

\subsection{Command-line}

The program is split into two halves: a client and a server.

\subsubsection{Server}

To run the server, simply run

    ./server

It will sit in the foreground and listen for requests from the client.
By default, the server listens on port 24992.
To listen on another port, pass the port number as the first argument to the server.

    ./server 8000

\subsubsection{Client}

The client is a little more interesting.
It has a number of subcommands;
to see the full list, give it the `-h` option (or any other invalid option

    \% ./client -h
    ./client: invalid option -- 'h'
    Usage: client [-p port] <command> [args]
    Commands:
        client setup [files...]
        client search <word>
        client add <fileid> [words...]
        client addfile <filename>
        client delete <fileid> [words...]

For any of these commands to work, the server must already be running.
If the server is not running, the client will hang until the server is started.

\subsubsubsection{Setup}

The setup command creates an encrypted search index

    client setup [files...]

For example,

    ./client setup *.cpp

will initialize the search index with all cpp files in the current directory.
If

There is also a hidden `setuplist` command, which reads the list of files from another file
instead of the command line.

For example,

    ls *.cpp >filelist
    ./client setuplist filelist

This does the same thing as the previous setup command, but can be used even if
the number of files is too large to fit on the command line.

\subsubsubsection{Search}

The search command searches for a keyword in the encrypted index.
It prints out a list of matching files.

    client search [keyword]

For example,

    \% ./client search int
    info: loaded client state
    got response
    int: bench.cpp
    int: client.cpp
    int: client_test.cpp
    int: DSSEClient.cpp
    int: DSSE.cpp
    int: DSSEServer.cpp
    int: fail.cpp
    int: merged-client.cpp
    int: server.cpp
    int: speed.cpp
    int: storage.cpp

\subsubsubsection{Add}

The add command adds one or more keywords to a file.
To use the add command, you need to know the id of the file you want to modify.

    client add <fileid> [keywords]

For example,

    ./client add 1 hello

Now searching for `hello' will return the file with id number 1.

    \% ./client search hello
    info: loaded client state
    got response
    hello: bench.cpp

There is also a variant of the add command called addfile.
The addfile command adds an entire new file to the search index.

    \% ./client addfile DSSE.h

\subsubsubsection{Delete}

The delete command deletes one or more keywords from a file.
It works similarly to the add command.
To use the delete command, you need to know the id of the file you want to modify.

    client delete <fileid> [keywords]

For example,

    ./client delete 1 hello

\subsection{Demo app}

The demo app lives in the `demo_app` branch.
It is not part of the main code base.



\section{Hacking}

The top level of the repository is layed out as follows.

    Benchmarks/         benchmark results
    Meeting_Notes/      meeting notes throughout the term
    Poster/             our expo poster
    Term_PowerPoints/   powerpoint slides from our progress reports
    demo/               (demo_app branch only) this is where the demo app lives
    reports/            capstone reports and other documents
    src/                the main source code
    third_party/        copies of third-party libraries that we depend on



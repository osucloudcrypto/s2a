% document include the answer to the conclusions and reflections questions

\chapter{Reflections}

% Each member will answer these questions ... no BS 
% Group Member 1
\subsection{Scott Russell}

\subsubsection{What technical information did you learn?}
There was allot of research done initially for this project. Security itself is a growing industry and to be able to maintain proficiencies we had to delve into some very modern crypto concepts. This is where the bulk of the focus went in the fall term, research. Winter term we were able to focus mostly on the implementation of the core DSSE scheme proposed by David Cash. His paper was the basis for our scheme, and our work as a whole. The concept of search-able encryption is extremely interesting to me as a security applied option and is considers some of the cutting edge technology in the security field. Being able to work with this new and research heavy work was very inspiring to future prospects as a graduate. Having this hands on experience has prepared me well for my future career as a Cyberwarfare officer.
\subsubsection{What non-technical information did you learn?}
A vast majority of this class was dealing with communication, coordination and work flow and supervision. It seemed much more toned toward a real world job situation. There was allot of scheduling that needed to be done that was difficult when trying to align free time between myself, Andrew Ekstedt, Scott Merrill, Andrew Emmott, Attila Yavuz and Thang Hoang. Trying to find a time that worked with all of us, or most of us, to meet and discuss progress and touch base, was difficult. This is akin to how scheduling can be in the work force where you have clients and teammates working remotely from across the world. These skills will be very useful in transitioning after graduation and I'm glad that this class brought up these important skills.

\subsubsection{What have you learned about project work?}
Scheduling is extremely difficult. With having cross town group mates, people getting sick, and in general life happening we weren't able to meet as a full team as often as I would of liked to. Working with a client was also an new experience for me. Instead of having a specified due date for each part of the project it was difficult for me to stay motivated. Therefore, I utilized mini-due dates that I would personally meet for work load. For example, "I will have this benchmark data done by Monday." Therefore I could use this small goals as plans in my OneNote to be able to continue progress week to week. As the term wrapped up I found it easier to stay focused as big final project deadlines began to loom closer. Now that we are finishing up the project I found that the entire experience of trying to manage a project is much less programming and much more communication than I initially thought.
\subsubsection{What have you learned about project management?}
Similar to project work managing the project has allot to do with setting and meeting short term goals to reach long term successes. You don't write a 50+ page Senior Capstone paper over night, well at least I hope no team did that! We spent the last 9 months working toward this goal. With small documents, each one adding to the final report, each page of code getting us closer to meeting requirements descriptions. Looking back at what we did I am proud of our team and how much we were able to accomplish in this time frame. If you had sat me down day one of capstone and told me I would be writing a 70 page paper by the end of the term with my team I would have given up. But because it was small increments, small successes and bit by bit we were able to come together to build something worth while.
\subsubsection{What have you learned about working in teams?}
Leadership and teamwork skills are paramount to success in a group environment. Being in the ROTC program here in campus we get drilled allot about the important to succeed or fail as a group. You may be an all star programmer, communicator and timely at work flow. But if you do it at the detriment to the rest of your team and leave them behind you will end up failing together. An analogy I like to refer to in this regard is "You have to step up to perform, but you are a single person on the team, be there supporting the rest at the plate." You have to step up to help your team complete their tasks, and they will do the same. It's a symbiotic relation.
\subsubsection{If you could do it all over, what would you do differently?}
I would have spent more time in the planning and preparing stage. This section of the work turned out to be extremely important in planning how we would be implementing different chunks of our project throughout the course of the assignment. We had some problems with falling behind our initial gantt chart that we prepared in fall term. We were able to adjust our project requirements to match the new scope and goal of capstone. Approval of this part with Attila turned out simple as he was very dynamic and adaptive to change as long as we maintained forward progress throughout the year.

% Group Member 2
\subsection{Group Member 2}

\subsubsection{What technical information did you learn?}
This project taught me a lot about cryptographic primitives, specifically dynamic searchable encryption. This was new to me and there was a large learning curve at the beginning of the project as much of the cryptography that was used I had not learned about before. This changed over the course of the years as I took more classes that address this.

\subsubsection{What non-technical information did you learn?}
Another major learning point from this project was how to work no a larger scale project than what OSU typically offers in other classes. The scope of this project, being over 3 terms, had us creating something much more complex than we normally can. Additionally, working as a group and learning how to break the project into different parts was new to me and I learned a lot from the experience.

\subsubsection{What have you learned about project work?}
Capstone did a great job teaching us about project work. One major emphasis that they placed upon us was to take good notes. We took notes for all meetings with out TA, Client, and during class. In addition to taking notes during meetings, we also took notes while working on our project. Making documentation for what we did, issues we ran across, and plans for the project going forward was a great learning experience.

\subsubsection{What have you learned about project management?}
This course had a heavy focus on project management. From the initial design to how to break down the project into digestible chunks was essential to our success. With a project on this scale it can be difficult to manage our time and goals appropriately to be successful and have a complete product by the time expo came around. This class does a great job of teaching us these points by having us experience these challenges first hand.

\subsubsection{What have you learned about working in teams?}
Working with others can always be a challenge. You need to be able to coordinate schedules, communicate effectively, work as a team, and learn about others strengths and weaknesses. By being a part of a project for the last 9 months, I was able to learn a lot about how to work well with a team. I thought this experience was one of the most valuable things that this course offers.

\subsubsection{If you could do it all over, what would you do differently?}
Looking back, now that the project is finished. The one thing I think I would have done differently would be to work with our client early to establish realistic expectations. We struggled in the beginning of the project because it had too many requirements than would be realistic for us to complete as a capstone project.

% Group Member 3
\subsection{Andrew Ekstedt}
\subsubsection{What technical information did you learn?}

Primarily, I learned about searchable encryption. I had never encountered the concept before this project, and now, having spent nine months working on it, I feel like I have a good knowledge of some of the ways it can be used, and some of its limitations. 

I also learned more about working in C++, which I didn't have much experience in prior to this capstone project. I'm still of the opinion that C++ is a poorly designed language - there is a lot of complexity and odd corner cases lurking to snare the uninitiated. I took some of the lessons learned about writing C++ and applied them to my class projects this term.

\subsubsection{What non-technical information did you learn?}


\subsubsection{What have you learned about project work?}
Don't depend on anyone.
% ...

\subsubsection{What have you learned about project management?}
I'm bad at project management.
% ...

\subsubsection{What have you learned about working in teams?}


\subsubsection{If you could do it all over, what would you do differently?}
